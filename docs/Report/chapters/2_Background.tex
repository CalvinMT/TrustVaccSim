\chapter{Background}

\section{Vaccines}

In 1796, Edward Jenner tested the world's first vaccine after discovering the link between two similar diseases (cowpox and smallpox), one less harmful than the other \cite{clem_fundamentals_2011}. Since then, as technology evolved, vaccination is commonly used around the world to prevent lives from being tormented by various diseases transmitted by different pathogens of various origins \cite{woolhouse_host_2005}. Basically, a vaccine trains the immunity system on dummy cells or proteins to battle a certain pathogen before infection by the real pathogen itself.

\subsection{Development and approval process}
\label{vaccine_development_approval_process}
Before a vaccine gains approval to be sold and used by the public, it must first undergo multiple development and testing stages, in addition to a scientific evaluation. First, companies ensure the pharmaceutical quality of the vaccine through small scale studies. Then, the vaccine is tested in non-clinical trials either in using in vitro (e.g., in cell cultures) or in vivo (on animals) studies. Finally, before scientific evaluation and authorisation leading to large-scale production, the vaccine undergoes three phases of clinical trials on human volunteers. This last stage enables the understanding of the vaccine's effectiveness, safety, side effects and optimal dosage \cite{ema_covid-19_2020}.

\subsection{Previous controversies}
Wakefield's vaccine-autism connection of measles, mumps and rubella (MMR) vaccination is the most recent and widely known vaccine controversy \cite{flaherty_vaccine-autism_2011}. Between the publication of his second paper in 2000 and its retraction ten years later, public media involved itself in the story, making the population doubt about the effectiveness and the purpose of vaccines in general. During this time, it was reported that measles and mumps cases in the United Kingdom increased as MMR vaccination rates decreased \cite{uk_hsa_measles_2022, phe_dhsd_vaccination_rates_2011}.

Other vaccine controversies, such as the one on the hepatitis B vaccine, which was wrongly thought to give multiple sclerosis \cite{nauche_3043_2001, giacometti_derives_2001, favereau_mensonges_canal_plus_liberation_2001, bursaux_mensonges_le_monde_2001, royo_vaccin_2009}, demonstrate how easily and quickly plainly wrong information, false claims and accusations can reduce a population's trust in vaccines \cite{francois_vaccine_2005}.



\section{SARS-CoV-2 and COVID-19}
\label{sars-cov-2_covid-19}

Coronavirus Disease 2019 (COVID-19) is an infectious illness caused by Severe Acute Respiratory Syndrome Coronavirus 2 (SARS-CoV-2), a virus which first got attention in China in late 2019 \cite{who_origin_sars-cov-2_2020}. The World Health Organisation (WHO) first expressed concerns over COVID-19 as an unknown pneumonia resembling a previous epidemic viral disease known as SARS (SARS-CoV or SARS-CoV-1), which also emerged from China in 2002 \cite{who_sars_nodate, stadler_sars_2003, chan-yeung_sars_2003}. Further concerns from the WHO and the rapid spread of SARS-CoV-2 made governments around the world apply measures to avoid further spreading the virus, or to at least slow it down.

Measures varied between countries, but even those which made efforts could only delay the propagation of the virus inside their borders. Within two to three months, the virus, along with its disease, spread globally. And within the following months, the disease caused hundred of thousands of deaths \cite{who_dashboard_nodate}.
Governments' actions to reduce the amount of severely ill patients in hospitals and to overcome the lack of medical staff, medical beds and respirators by the means of quarantines, curfews, lockdowns and sanitary obligations did not go without social and economic impacts \cite{mofijur_impact_2021, brodeur_literature_2021}.



\section{COVID-19 vaccines controversy}

About one year after the detection of SARS-CoV-2, the first dose of vaccine was administered to the public \cite{nhs_first_vaccination}. Now in 2022, among the 38 approved vaccines, 10 of them were granted emergency use by the WHO \cite{who_covid-19_vaccines_approved_2022, who_covid-19_vaccines_2022}. Out of the most commonly used COVID-19 vaccines, two of them are mRNA vaccines, which are the first two of their kind to be commercially distributed for human application \cite{uk-dhsc_pfizer_vaccine_2020, dolgin_tangled_2021}. Although human clinical trials for mRNA vaccines already took place in 2008 \cite{weide_results_2008, sahin_mrna-based_2014}, some misleading information over how these vaccines got granted emergency commercial use spread fear among the population, thinking that the development of these vaccines, at first, then all of them, was rushed and did not undergo the usual approval process (see section \ref{vaccine_development_approval_process}) \cite{uuhc_covid-19_vaccines_2022, care_is_2021}.
Some studies collected and mapped trust in the safety of vaccines in most countries world-wide \cite{figueiredo_mapping_2020, sallam_global_2022}. Comparing this data of countries' trust in vaccines with countries' vaccination rate \cite{mathieu_global_2021, ritchie_coronavirus_2020} makes it possible to see that countries with a low trust in vaccines (France, Japan, South Korea) have a slow vaccine intake start compared to other countries.



\section{Trust}
\label{trust}

Trust is the feeling of integrity in the action and speech of others \footnotemark[1]{}\footnotemark[2]{}. Actions shape the world in a way that only what is done may be proven, and speech may be true only if it reflects how the world is shaped. People therefore put their trust in what may be proven within the reach they have on the world. If an action or a speech is proven to be true, one will trust their actor.

\footnotetext[1]{Trust: 1. to believe that someone is good and honest and will not harm you [{\textellipsis}] -- Cambridge University Press -- 2022 (\url{https://dictionary.cambridge.org/dictionary/english/trust})}
\footnotetext[2]{Trust: 1. the belief that somebody/something is good, sincere, honest, etc. and will not try to harm or trick you; 2. the belief that something is true or correct or that you can rely on it -- Oxford University Press -- 2022 (\url{https://www.oxfordlearnersdictionaries.com/definition/english/trust_1})}

\pagebreak

\subsection{Interpersonal trust}

Once one has gained trust, people will be more willing to believe this trusted acquaintance than strangers from which no action resulting in trust or distrust was experienced \cite{guha_propagation_2004}. A trustworthy individual may then influence the trust that their acquaintances have in others \cite{gibbs_review_1990}. However, if one is lacking trust, they will be less influenced than one showing trust \cite{cvetkovich_new_2002}.
Additionally, if one wants to trust another, then it is of interest for the former to make the latter trust them \cite{gibbs_review_1990}. This brings mutual trust, where two individuals trust each other. Relations of mutual trust, being a system of positive feedback, is an effective system for beliefs to propagate \cite{guha_propagation_2004}.

\subsection{Institutional trust}

Trust can be put into a person or into a system, such as a government. In order for a government to trust its citizens to obey the law and follow exceptional rules and measures (e.g., curfews, lockdowns or quarantines), it must make sure that its citizens find it sufficiently trustworthy and reliable.
A trustworthy government should be able to keep its commitments by displaying its competence to put non-arbitrary policies into effect. The trustworthiness of a government is also linked to the morality of office holders. In addition to becoming trustworthy and in order to gain trust, a government must make sure its citizens possess sufficient knowledge to understand and believe that its actions are taken for their best interest. If not, citizens trust will decrease as they would think, through their own perception, that social problems are getting worse or left unresolved \cite{levi_political_2000}.

Government officials appear more trustworthy when they trust their citizens, and citizens that find their government to be trustworthy are more likely to comply to rules and regulations \cite{levi_political_2000, devine_trust_2021}. Similarly, trust between citizens is strongly influenced by the trust they each have towards the same government. In other words, a trustworthy government can, if trusted, help maintain trust among its citizens \cite{levi_political_2000}.

\subsection{Information influence over trust}

Trust can also be influenced by the nature of collected information (positive or negative) while being in a state of trust or distrust \cite{cvetkovich_new_2002}, as displayed in table \ref{tab:trust_info_influence}. If the information is positive and that the one receiving the information shows trust, then a reinforcement of its beliefs takes place as well as an increase of its trust. On the other hand, if the information is still positive but that the receiver shows distrust, then the information is viewed as less important than it is, although its trust will slightly increase. On the opposite side, if the nature of the information is negative and that the person receiving the information carries trust, then the information will be neglected by this person despite the slight decrease in its trust. But if the information is negative and that the receiver of the information already bears distrust, then a reinforcement of its existing beliefs occurs at the same time as an increase in its distrust.

In other words, if perceived information follows one's beliefs, a reinforcement of those beliefs takes place. If not, the information is discounted.

\begin{table}[hbt]
    \centering
    \begin{tabular}{l|ll|ll}
        \cline{2-3}
        \multirow{2}{*}{} & \multicolumn{2}{c|}{Information type} &  &  \\ \cline{2-3}
         & \multicolumn{1}{c|}{Positive} & \multicolumn{1}{c|}{Negative} &  &  \\ \cline{1-3}
        \multicolumn{1}{|l|}{Trust} & \multicolumn{1}{l|}
            {\begin{tabular}[c]{@{}l@{}}
                - Reinforcement of existing beliefs\\
                - Increase in trust
            \end{tabular}}
            & \begin{tabular}[c]{@{}l@{}}
                - Discounting of the information\\
                - Slight decrease in trust, if any
            \end{tabular} &  &  \\ \cline{1-3}
        \multicolumn{1}{|l|}{Distrust} & \multicolumn{1}{l|}
            {\begin{tabular}[c]{@{}l@{}}
                - Discounting of the information\\
                - Slight increase in trust, if any
            \end{tabular}}
            & \begin{tabular}[c]{@{}l@{}}
                - Reinforcement of existing beliefs\\
                - Increase in distrust
            \end{tabular} &  &  \\ \cline{1-3}
    \end{tabular}
    \caption{Influence of information types on trust. \label{tab:trust_info_influence}}
\end{table}

\subsection{Trust during the COVID-19 pandemic}
\label{trust_covid19}

During the COVID-19 pandemic (see section \ref{sars-cov-2_covid-19}), governments pushed measures in order to contain the virus and slow down its spread into and within countries. Measures went from mandatory mask wear and hand sanitising to curfews and lockdowns. Some people had difficulty respecting these measures by lack of trust in the authorities. Eventually, others got tired of the measures which made them, in turn, lose trust in authorities and made respecting rules and guidelines a harder task \cite{strandberg2020coronavirus, goldstein2021lockdown}.
Additionally, information inconsistency and lack in their clarity emphasised the population's hesitancy. As example, France faced several misleading successions of information on various topics such as mask wear, virus transmission in children and lockdowns
\cite{gautreau_informations_contradictoires_2020, mazoue_masque_inutile_a_obligatoire_2020}.
Furthermore, false information, may it be misinformation or disinformation, spread mainly across social media and lowered trust among the population \cite{oecd_transparency_disinformation_2020, enria_trust_survey_2021, erku_misinformation_2021, abd-alrazaq_top_2020}.

\subsection{Trust in COVID-19 vaccines}

Once COVID-19 vaccines were being distributed, people within the population were found hesitant towards them, or even completely opposed to them. It has been studied and shown that a low level of trust among the population is linked to low vaccination rates \cite{hornsey_psychological_2018, soares_factors_2021, forman_covid-19_2021}. Thus, the more a population trusts vaccines and institutions, the higher the vaccination rates will be.
Although this is true, increasing a population's trust is no easy task. It cannot increase simply by giving out information, such as:
\begin{itemize}
    \item \textquote{COVID-19: unvaccinated represent 56\% of entries in critical care and 42\% of deaths} \cite{lci_covid-19_2022} (Does this mean that vaccinated --- with or without boosters? --- account for 58\% of deaths? Should we conclude that vaccination is worse than useless? Or is it that unvaccinated represent 42\% of deaths among all other diseases and reasons to die?);
    \item \textquote{COVID-19: Patrick Pelloux indicates that "75\% of sick patients in intensive care units are unvaccinated"}\cite{bfmtv_tweet_reanimation_2021} (It could be that 25\% of sick patients in intensive care units left are vaccinated. But how could one possibly interpret this information correctly if it is not put against the proportion of vaccinated in the population?).
\end{itemize}
As for the inconsistency in information mentioned in section \ref{trust_covid19}, information on COVID-19 vaccines is not as clear as it could be, unavoidably leading to its misinterpretation \cite{machingaidze_understanding_2021, tentori_misunderstanding_2021}.
Hopefully, some media had the courage to explain in detail how there can be more vaccinated than unvaccinated patients in hospitals \cite{lesoir_infographie_hospitalises_vaccines_2021, francetvinfo_morts_vaccines_2021, lardeur_covid-19_2021}.

\pagebreak

\vskip\baselineskip
This constant misrepresentation of information is a problem when the objective is to increase people's trust in vaccines, science and institutions. Moreover, it shows that not enough people question the information given to them. Those are the reasons that motivated the present work to take place with the goals:
\begin{enumerate}
    \item to raise awareness in media and personalities of their responsibility to be as precise as possible in their speech and writing when giving out information;
    \item to educate people about the importance to question gathered information in order to avoid misinterpretation and its outcome: concerning the present work, avoidable deaths.
\end{enumerate}
