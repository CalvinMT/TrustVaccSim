\chapter{Discussion}

To build an educational simulation with an agent-based model of an epidemic with vaccination and trust, a lot of knowledge had to be gathered through research within various disciplines. Doing so allowed the implementation of an SIAHRD epidemiological model and helped with the application of a vaccination status, as well as with the addition of a trust human behaviour attribute and its methods.

Although the results look concluding, logical and seem close enough to realistic for an educational simulation, it is hard to find any data that could validate them. One way that the results could be validated would be to have them examined by epidemiologists and psychologists.
Additionally, and as noticeable in section \ref{results_analysis}, various observations can be made. Thus, there is a need to run a proper experimentation with participants in order to see if the different expected observations are understood by various members of the public.



\section{Future plans}

Not everything thought of during the time taken to focus on the present work could be implemented. The subject is as deep as the researcher wants it to be and data to attain the desired goal is scattered between scientific articles, news journals, independent websites and social media. However, here are a few ideas to continue on the course of this work.

One could consider making the environment of the simulation a town or a city with households \cite{grignard_gama_2013}. Influence within those households would make it possible to view greater changes in trust and vaccination in some parts of the environment than others.
Additionally, hospital visits would be organised differently, as only members of a same household would visit hospitalised agents.

Incorporating age groups would allow implementing family influence over trust. This could be important as children are more than often influenced by their parents \cite{sheehan_trust_2020}. Additionally, younger generations are more inclined to show refusal and delay in vaccine intake \cite{soares_factors_2021}.

Extending the model with different information source types (government documents, medical recommendations, or social media) could influence the trust level of agents in more diverse ways, as it has been reported that, younger generations tend to get their information on social media more than other generations \cite{okeeffe_impact_2011, roozenbeek_susceptibility_2020}.



\section{End note}

Even if vaccination becomes mandatory, if a population does not show trust, may it be in vaccines in general or in the government, people will find ways to get around policies they do not feel trustworthy. Reports of people getting fake vaccine passes and health care assistants providing counterfeit vaccine injections were making news during the SARS-CoV-2 epidemic \cite{davies_fake_covid_passes_2021, cnn_germany_fake_vaccines_2021, tondo_italy_fake_vaccines_2022}.
To prevent such behaviours, political and medical authorities should constantly double their efforts in correcting misinformation and countering disinformation \cite{abd-alrazaq_top_2020}. However, increase of trust should be sought prior to these efforts, which might even prevent them from being required.
In parallel, citizens must appreciate the importance of questioning gathered information with conscious humility to avoid any misinterpretation which could lead, as the present work attempted to show, to irreversible and tragic events.
